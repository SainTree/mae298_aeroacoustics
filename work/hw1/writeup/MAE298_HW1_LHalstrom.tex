
%%% use twocolumn and 10pt options with the asme2ej format
\documentclass[twocolumn,10pt]{asme2ej}
%% The class has several options
%  onecolumn/twocolumn - format for one or two columns per page
%  10pt/11pt/12pt - use 10, 11, or 12 point font
%  oneside/twoside - format for oneside/twosided printing
%  final/draft - format for final/draft copy
%  cleanfoot - take out copyright info in footer leave page number
%  cleanhead - take out the conference banner on the title page
%  titlepage/notitlepage - put in titlepage or leave out titlepage
%
%% The default is oneside, onecolumn, 10pt, final


\usepackage{epsfig} %% for loading postscript figures
\usepackage{listings}
\usepackage{amsmath}
\usepackage{graphicx}
\usepackage{grffile}
\usepackage{pdfpages}
\usepackage{algpseudocode}
\usepackage{courier}
\usepackage{tikz}
\newcommand*\circled[1]{\tikz[baseline=(char.base)]{
            \node[shape=circle,draw,inner sep=2pt] (char) {#1};}}
%\usepackage{multicol}

\usepackage{caption}          %caption graphics ('\captionof')
\usepackage{framed}           %allows shaded text
\usepackage[T1]{fontenc}      %allows escaping specific charaters like '_'





% % Custom colors
% \usepackage{color}
% \usepackage{listings}
% \usepackage{framed}
% \usepackage{caption}
% \usepackage{bm}
% \captionsetup[lstlisting]{font={small,tt}}

% \definecolor{mygreen}{rgb}{0,0.6,0}
% \definecolor{mygray}{rgb}{0.5,0.5,0.5}
% \definecolor{mymauve}{rgb}{0.58,0,0.82}

% \lstset{ %
%   backgroundcolor=\color{white},   % choose the background color; you must add \usepackage{color} or \usepackage{xcolor}
%   basicstyle=\ttfamily\footnotesize, % the size of the fonts that are used for the code
%   breakatwhitespace=false,         % sets if automatic breaks should only happen at whitespace
%   % breaklines=true,                 % sets automatic line breaking
%   captionpos=b,                    % sets the caption-position to bottom
%   commentstyle=\color{mygreen},    % comment style
%   deletekeywords={...},            % if you want to delete keywords from the given language
%   escapeinside={\%*}{*)},          % if you want to add LaTeX within your code
%   extendedchars=true,              % lets you use non-ASCII characters; for 8-bits encodings only, does not work with UTF-8
%   frame=single,                    % adds a frame around the code
%   keepspaces=true,                 % keeps spaces in text, useful for keeping indentation of code (possibly needs columns=flexible)
%   columns=flexible,
%   keywordstyle=\color{blue},       % keyword style
%   language=Python,                 % the language of the code
%   morekeywords={*,...},            % if you want to add more keywords to the set
%   numbers=left,                    % where to put the line-numbers; possible values are (none, left, right)
%   numbersep=5pt,                   % how far the line-numbers are from the code
%   numberstyle=\tiny\color{mygray}, % the style that is used for the line-numbers
%   rulecolor=\color{black},         % if not set, the frame-color may be changed on line-breaks within not-black text (e.g. comments (green here))
%   showspaces=false,                % show spaces everywhere adding particular underscores; it overrides 'showstringspaces'
%   showstringspaces=false,          % underline spaces within strings only
%   showtabs=false,                  % show tabs within strings adding particular underscores
%   stepnumber=1,                    % the step between two line-numbers. If it's 1, each line will be numbered
%   stringstyle=\color{mymauve},     % string literal style
%   tabsize=4,                       % sets default tabsize to 2 spaces
% }



%%%%%%%%%%%%%%%%%%%%%%%%%%%%%%%%%%%%%%%%%%%%%%%
%CODE LISTING SYNTAX COLORING
\usepackage{color}     %make custom colors for syntax coloring
\usepackage{listings}  %allows code listings
\usepackage{textcomp}  %allows apostrophes to be straight (unidirectional)
% \usepackage{framed}
% \usepackage{caption}
\usepackage{bm}
\usepackage{tcolorbox}        %use for code listing color definitions
\tcbuselibrary{listings}      %allow color defenitions in code listings
\tcbuselibrary{breakable}     %allow tccolorboxes across page breaks
\tcbuselibrary{most}     %allow tccolorboxes across page breaks

\captionsetup[lstlisting]{font={small,tt}} %setup caption style

%Custom Colors
\definecolor{mygreen}{rgb}{0,0.6,0}
\definecolor{mygray}{rgb}{0.5,0.5,0.5}
\definecolor{mymauve}{rgb}{0.58,0,0.82}

\definecolor{sublimeblack}{HTML}{272822}
\definecolor{sublimered}{HTML}{F92672}
\definecolor{sublimeblue}{HTML}{66D9EF}
\definecolor{sublimeyellow}{HTML}{E6DB74}
\definecolor{sublimegrey}{HTML}{75715E}
\definecolor{sublimegreen}{HTML}{66CC33}
\definecolor{sublimeorange}{HTML}{FD971F}
\definecolor{sublimepurple}{HTML}{AE81FF}


%White or Black background toggle
\newcommand\whiteback{0}
\ifnum\whiteback=1%
  %White Background
  \newcommand\backclr{white}
  \newcommand\txtclr{\color{black}}
  \newcommand\txtupclr{black} %same as text color, used elsewhere
  \newcommand\comclr{\color{sublimegrey}}
  % \newcommand\comclr{\color{mygreen}}
  \newcommand\keyclr{\color{sublimeblue}}
  % \newcommand\keyclr{\color{blue}}
  \newcommand\ndkeyclr{\color{sublimered}}
  \newcommand\strclr{\color{mygreen}}
  % \newcommand\strclr{\color{sublimeyellow}}
  \newcommand\frameon{single} %single-frame box around code

\else
  %Black Background (like SublimeText)
  \newcommand\backclr{sublimeblack}
  \newcommand\txtclr{\color{white}} %text color
  \newcommand\txtupclr{white} %same as text color, used elsewhere
  \newcommand\comclr{\color{sublimegrey}}
  \newcommand\keyclr{\color{sublimeblue}}
  \newcommand\ndkeyclr{\color{sublimered}}
  \newcommand\strclr{\color{sublimeyellow}}
  \newcommand\frameon{false} %no frame for black background
\fi


%Listing Style
% \newcommand\lstfontsize{\scriptsize}
\newcommand\lstfontsize{\small}
\lstdefinestyle{mystyle}{%
  % backgroundcolor=\backclr,   % choose the background color; you must add \usepackage{color} or \usepackage{xcolor}
  basicstyle=\ttfamily\txtclr\lstfontsize, % the SIZE OF THE FONTS that are used for the code
  breakatwhitespace=false,         % sets if automatic breaks should only happen at whitespace
  % breaklines=true,                 % sets automatic line breaking
  captionpos=b,                    % sets the caption-position to bottom
  commentstyle=\comclr,            % comment color
  deletekeywords={...},            % if you want to delete keywords from the given language
  escapeinside={\%*}{*)},          % if you want to add LaTeX within your code
  extendedchars=true,              % lets you use non-ASCII characters; for 8-bits encodings only, does not work with UTF-8
  frame=\frameon,                  % adds a frame around the code
  keepspaces=true,                 % keeps spaces in text, useful for keeping indentation of code (possibly needs columns=flexible)
  columns=flexible,
  otherkeywords={zip,enumerate,True,False,None,...},  %add words to be highlighted
  keywordstyle=\keyclr,            % keyword (e.g. 'print') color
  language=Python,                 % the language of the code
  upquote=true,                    % make apostrophes straight (unidirectional)
  alsoletter={<>=-+*/!},            % to avoid coloring operators when they're not
  ndkeywords={=,+,-,*,**,/,+=,*=,-=,/=,<=,>=,==,!=,<,>,... },  % operator keywords
  ndkeywordstyle=\ndkeyclr,         % style of operator keywords
  numbers=left,                    % where to put the line-numbers; possible values are (none, left, right)
  numbersep=5pt,                   % how far the line-numbers are from the code
  numberstyle=\tiny\color{mygray}, % line-number style: size, color
  rulecolor=\color{black},         % if not set, the frame-color may be changed on line-breaks within not-black text (e.g. comments (green here))
  showspaces=false,                % show spaces everywhere adding particular underscores; it overrides 'showstringspaces'
  showstringspaces=false,          % underline spaces within strings only
  showtabs=false,                  % show tabs within strings adding particular underscores
  stepnumber=1,                    % the step between two line-numbers. If it's 1, each line will be numbered
  stringstyle=\strclr,             % string color
  tabsize=4,                       % sets default tabsize
}


%Function that calls listing of specific file with custom coloring
  %first input is filename to list
  %next two inputs are start and end line numbers
    %(use 0 and large number for all lines)
\newcommand\mylisting[3]{
  \tcbinputlisting{
        listing file=#1,
        breakable,
        nobeforeafter,
        arc=0pt,
        top=0mm,
        bottom=0mm,
        left=0mm,
        right=0mm,
        boxrule=0pt,
        colback=\backclr, %background color (continue accross page breaks)
        colupper=\txtupclr, %basic text color (continue accross page breaks)
        listing only,
        listing options={
                            style=mystyle,            %use custom style
                            firstline=#2,lastline=#3, %lines of code to use
                            firstnumber=#2,           %same numbering as script
                        },
        % hbox %Turning this on prevents page breaks
      }
}



%%%%%%%%%%%%%%%%%%%%%%%%%%%%%%%%%%%%%%%%%%%%%%%%%%%%%%%%%%%%%%%%%%%%%%%%
%%% MAIN DOCUMENT %%%%%%%%%%%%%%%%%%%%%%%%%%%%%%%%%%%%%%%%%%%%%%%%%%%%%%
%%%%%%%%%%%%%%%%%%%%%%%%%%%%%%%%%%%%%%%%%%%%%%%%%%%%%%%%%%%%%%%%%%%%%%%%

\title{MAE 298 -- Homework 1\\Computation of Sound Pressure Level\\and Octave Band Spectrum}

%%% first author
\author{Logan Halstrom
    \affiliation{
	PhD Graduate Student Researcher\\
	Center for Human/Robot/Vehicle Integration and Performance\\
	Department of Mechanical and Aerospace Engineering\\
	University of California, Davis\\
	% Davis, California 95616\\
    Email: ldhalstrom@ucdavis.edu
    }
}

\begin{document}
\maketitle


\newcommand\pictype{png}  %set filetype of images to read in
\newcommand\picdir{../Plots} %directory images are located in



%%%%%%%%%%%%%%%%%%%%%%%%%%%%%%%%%%%%%%%%%%%%%%%%%%%%%%%%%%%%%%%%%%%%%%
\section{Introduction}
%%%%%%%%%%%%%%%%%%%%%%%%%%%%%%%%%%%%%%%%%%%%%%%%%%%%%%%%%%%%%%%%%%%%%%%%

Give overview of homework and background concepts


%%%%%%%%%%%%%%%%%%%%%%%%%%%%%%%%%%%%%%%%%%%%%%%%%%%%%%%%%%%%%%%%%%%%%%%%
\section{Read Data}
%%%%%%%%%%%%%%%%%%%%%%%%%%%%%%%%%%%%%%%%%%%%%%%%%%%%%%%%%%%%%%%%%%%%%%%%

list functions used to read data, how python/matlab compare.

%%\vspace{-2em}
\begin{figure}[htb]
\begin{center}
\includegraphics[width=0.5\textwidth]{\picdir/1_1_Pressure.\pictype}
\caption{Recorded sonic boom shockwave pressure time history in characteristic high-low pressure N-wave shape (Zero-pressure from recording start to initial shock)}
\label{pic_Nwave}
\end{center}
\end{figure}
%%\vspace{-2em}



%%%%%%%%%%%%%%%%%%%%%%%%%%%%%%%%%%%%%%%%%%%%%%%%%%%%%%%%%%%%%%%%%%%%%%%%
\section{Frequency Domain}
%%%%%%%%%%%%%%%%%%%%%%%%%%%%%%%%%%%%%%%%%%%%%%%%%%%%%%%%%%%%%%%%%%%%%%%%

decompose into frequency domain with FFT


%%%%%%%%%%%%%%%%%%%%%%%%%%%%%%%%%%%%%%%%%%%%%%%%%%%%%%%%%%%%%%%%%%%%%%%%
\subsection{Power Spectral Density Decomposition}

power spectrum density decomposition stuff

%%\vspace{-2em}
\begin{figure}[htb]
\begin{center}
\includegraphics[width=0.5\textwidth]{\picdir/1_2_PowerSpec.\pictype}
\caption{Shockwave signal power spectral density as a function of frequency (All frequencies above 50Hz very low power)}
\label{pic_spec}
\end{center}
\end{figure}
%%\vspace{-2em}

%%%%%%%%%%%%%%%%%%%%%%%%%%%%%%%%%%%%%%%%%%%%%%%%%%%%%%%%%%%%%%%%%%%%%%%%
\subsection{Sound Pressure Level}

this is actually in the plot in the next section


%%%%%%%%%%%%%%%%%%%%%%%%%%%%%%%%%%%%%%%%%%%%%%%%%%%%%%%%%%%%%%%%%%%%%%%%
\section{Octave-Band Spectra} \label{sect_oct}
%%%%%%%%%%%%%%%%%%%%%%%%%%%%%%%%%%%%%%%%%%%%%%%%%%%%%%%%%%%%%%%%%%%%%%%%

%%\vspace{-2em}
\begin{figure}[htb]
\begin{center}
\includegraphics[width=0.5\textwidth]{\picdir/2_SPLF_all.\pictype}
\caption{Shockwave signal narrow-band, $\frac{1}{3}$ octave-band, and octave-band, with overall Sound Pressure Level reported in upper right}
\label{pic_oct}
\end{center}
\end{figure}
%%\vspace{-2em}




%%%%%%%%%%%%%%%%%%%%%%%%%%%%%%%%%%%%%%%%%%%%%%%%%%%%%%%%%%%%%%%%%%%%%%%%
\section{Conclusion}

conclude

%%%%%%%%%%%%%%%%%%%%%%%%%%%%%%%%%%%%%%%%%%%%%%%%%%%%%%%%%%%%%%%%%%%%%%%%
% \section{References}
% \begin{description}
% \item[] 1. Bogard, D.G., Teiderman, W.G. ``Burst detection with single-point velocity measurements'', \emph{Journal of Fluid Mechanics}, 162:389-413, 1986.
% \end{description}








%%%%%%%%%%%%%%%%%%%%%%%%%%%%%%%%%%%%%%%%%%%%%%%%%%%%%%%%%%%%%%%%%%%%%%%%
%APPENDIX (CODE LISTING)

% \clearpage
\onecolumn      %change to single column
\appendix       %%% starting appendix

%SETUP
\newcommand\scriptname{dummy}      %init code name variable
\captionsetup{width=0.8\linewidth} %set caption text width

\catcode`\_=12 %ESCAPE ALL UNDERSCORES IN TEXT AUTOMATICALLY*******************

%LISTINGS

\renewcommand\scriptname{hw1_00_process.py}
\section*{Appendix A: Data Processing Script}
\begin{center}

% \lstinputlisting[caption=text, language=Python]{../\scriptname}



\mylisting{../\scriptname}{0}{10000}
% \vspace{-0.2cm}
\captionof{lstlisting}{\emph{\scriptname} - Performs all primary data processing such as pressure signal input, power spectral density decomposition, and octave-band conversion and saves data to text files}




% \newboxarray{mylisting}
%   \tcbinputlisting{
%     listing file=../\scriptname,
%     colback=\backclr,
%     listing only,
%     hbox,
%     nobeforeafter,enhanced,
%     size=small,
%     store to box array=mylisting,
%     listing options={
%                       style=mystyle,            %use custom style
%                     }
%   }







\end{center}

\clearpage

% \renewcommand\scriptname{hw1_99_wrapper.py}
% \section*{Appendix A: Data Processing/Plotting Wrapper Script}
% \begin{center}
% \mylisting{../\scriptname}{0}{10000}
% % \vspace{-0.2cm}
% \captionof{lstlisting}{\emph{\scriptname} - Wrapper program that runs data processing and plotting scripts simultaneously}
% \end{center}

% \clearpage



% \section*{Appendix C: Parallel, Multi-Block Grid Decomposition Code}
% \lstinputlisting[caption=Grids are decomposed into blocks and mapped onto NPROCS processors and information pertaining to neighbors is stored using the GRIDMOD module, language=Fortran]{../modules.f90}

% \clearpage

% \section*{Appendix D: Multi-Block Plot3D Reader-Writer}
% \lstinputlisting[caption=Code for saving formatted multiblock PLOT3D solution files and reading restart files, language=Fortran]{../inout.f90}

% \clearpage




%%%%%%%%%%%%%%%%%%%%%%%%%%%%%%%%%%%%%%%%%%%%%%%%%%%%%%%%%%%%%%%%%%%%%%
%\clearpage


%%%%%%%%%%%%%%%%%%%%%%%%%%%%%%%%%%%%%%%%%%%%%%%%%%%%%%%%%%%%%%%%%%%%%%
% The bibliography is stored in an external database file
% in the BibTeX format (file_name.bib).  The bibliography is
% created by the following command and it will appear in this
% position in the document. You may, of course, create your
% own bibliography by using thebibliography environment as in
%
% \begin{thebibliography}{12}
% ...
% \bibitem{itemreference} D. E. Knudsen.
% {\em 1966 World Bnus Almanac.}
% {Permafrost Press, Novosibirsk.}
% ...
% \end{thebibliography}

% Here's where you specify the bibliography style file.
% The full file name for the bibliography style file
% used for an ASME paper is asmems4.bst.
%\bibliographystyle{asmems4}

% Here's where you specify the bibliography database file.
% The full file name of the bibliography database for this
% article is asme2e.bib. The name for your database is up
% to you.
%\bibliography{asme2e}

%%%%%%%%%%%%%%%%%%%%%%%%%%%%%%%%%%%%%%%%%%%%%%%%%%%%%%%%%%%%%%%%%%%%%%


\end{document}
