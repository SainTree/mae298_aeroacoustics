
%%% use twocolumn and 10pt options with the asme2ej format
\documentclass[twocolumn,10pt]{asme2ej}
%% The class has several options
%  onecolumn/twocolumn - format for one or two columns per page
%  10pt/11pt/12pt - use 10, 11, or 12 point font
%  oneside/twoside - format for oneside/twosided printing
%  final/draft - format for final/draft copy
%  cleanfoot - take out copyright info in footer leave page number
%  cleanhead - take out the conference banner on the title page
%  titlepage/notitlepage - put in titlepage or leave out titlepage
%
%% The default is oneside, onecolumn, 10pt, final


\usepackage{epsfig} %% for loading postscript figures
\usepackage{listings}
\usepackage{amsmath}
\usepackage{graphicx}
\usepackage{grffile}
\usepackage{pdfpages}
\usepackage{algpseudocode}
\usepackage{courier}
\usepackage{tikz}
\newcommand*\circled[1]{\tikz[baseline=(char.base)]{
            \node[shape=circle,draw,inner sep=2pt] (char) {#1};}}
%\usepackage{multicol}

\usepackage{caption}          %caption graphics ('\captionof')
\usepackage{framed}           %allows shaded text
\usepackage[T1]{fontenc}      %allows escaping specific charaters like '_'


%%%%%%%%%%%%%%%%%%%%%%%%%%%%%%%%%%%%%%%%%%%%%%%%%%%%%%
%HYPERLINKS
\usepackage{hyperref}     %insert hyperlines with '\url' or '\href'
\newcommand\linkcolor{blue} %variable to color url text

%%%%%%%%%%%%%%%%%%%%%%%%%%%%%%%%%%%%%%%%%%%%%%%
%CODE LISTING SYNTAX COLORING
\usepackage{color}     %make custom colors for syntax coloring
\usepackage{listings}  %allows code listings
\usepackage{textcomp}  %allows apostrophes to be straight (unidirectional)
% \usepackage{framed}
% \usepackage{caption}
\usepackage{bm}
\usepackage{tcolorbox}        %use for code listing color definitions
\tcbuselibrary{listings}      %allow color defenitions in code listings
\tcbuselibrary{breakable}     %allow tccolorboxes across page breaks
\tcbuselibrary{most}     %allow tccolorboxes across page breaks

\captionsetup[lstlisting]{font={small,tt}} %setup caption style

%Custom Colors
\definecolor{mygreen}{rgb}{0,0.6,0}
\definecolor{mygray}{rgb}{0.5,0.5,0.5}
\definecolor{mymauve}{rgb}{0.58,0,0.82}

\definecolor{sublimeblack}{HTML}{272822}
\definecolor{sublimered}{HTML}{F92672}
\definecolor{sublimeblue}{HTML}{66D9EF}
\definecolor{sublimeyellow}{HTML}{E6DB74}
\definecolor{sublimegrey}{HTML}{75715E}
\definecolor{sublimegreen}{HTML}{66CC33}
\definecolor{sublimeorange}{HTML}{FD971F}
\definecolor{sublimepurple}{HTML}{AE81FF}


%White or Black background toggle
\newcommand\whiteback{0}
\ifnum\whiteback=1%
  %White Background
  \newcommand\backclr{white}
  \newcommand\txtclr{\color{black}}
  \newcommand\txtupclr{black} %same as text color, used elsewhere
  \newcommand\comclr{\color{sublimegrey}}
  % \newcommand\comclr{\color{mygreen}}
  \newcommand\keyclr{\color{sublimeblue}}
  % \newcommand\keyclr{\color{blue}}
  \newcommand\ndkeyclr{\color{sublimered}}
  \newcommand\strclr{\color{mygreen}}
  % \newcommand\strclr{\color{sublimeyellow}}
  \newcommand\frameon{single} %single-frame box around code

\else
  %Black Background (like SublimeText)
  \newcommand\backclr{sublimeblack}
  \newcommand\txtclr{\color{white}} %text color
  \newcommand\txtupclr{white} %same as text color, used elsewhere
  \newcommand\comclr{\color{sublimegrey}}
  \newcommand\keyclr{\color{sublimeblue}}
  \newcommand\ndkeyclr{\color{sublimered}}
  \newcommand\strclr{\color{sublimeyellow}}
  \newcommand\frameon{false} %no frame for black background
\fi


%Listing Style
% \newcommand\lstfontsize{\scriptsize}
\newcommand\lstfontsize{\small}
\lstdefinestyle{mystyle}{%
  % backgroundcolor=\backclr,   % choose the background color; you must add \usepackage{color} or \usepackage{xcolor}
  basicstyle=\ttfamily\txtclr\lstfontsize, % the SIZE OF THE FONTS that are used for the code
  breakatwhitespace=false,         % sets if automatic breaks should only happen at whitespace
  % breaklines=true,                 % sets automatic line breaking
  captionpos=b,                    % sets the caption-position to bottom
  commentstyle=\comclr,            % comment color
  deletekeywords={...},            % if you want to delete keywords from the given language
  escapeinside={\%*}{*)},          % if you want to add LaTeX within your code
  extendedchars=true,              % lets you use non-ASCII characters; for 8-bits encodings only, does not work with UTF-8
  frame=\frameon,                  % adds a frame around the code
  keepspaces=true,                 % keeps spaces in text, useful for keeping indentation of code (possibly needs columns=flexible)
  columns=flexible,
  otherkeywords={zip,enumerate,True,False,None,...},  %add words to be highlighted
  keywordstyle=\keyclr,            % keyword (e.g. 'print') color
  language=Python,                 % the language of the code
  upquote=true,                    % make apostrophes straight (unidirectional)
  alsoletter={<>=-+*/!},            % to avoid coloring operators when they're not
  ndkeywords={=,+,-,*,**,/,+=,*=,-=,/=,<=,>=,==,!=,<,>,... },  % operator keywords
  ndkeywordstyle=\ndkeyclr,         % style of operator keywords
  numbers=left,                    % where to put the line-numbers; possible values are (none, left, right)
  numbersep=5pt,                   % how far the line-numbers are from the code
  numberstyle=\tiny\color{mygray}, % line-number style: size, color
  rulecolor=\color{black},         % if not set, the frame-color may be changed on line-breaks within not-black text (e.g. comments (green here))
  showspaces=false,                % show spaces everywhere adding particular underscores; it overrides 'showstringspaces'
  showstringspaces=false,          % underline spaces within strings only
  showtabs=false,                  % show tabs within strings adding particular underscores
  stepnumber=1,                    % the step between two line-numbers. If it's 1, each line will be numbered
  stringstyle=\strclr,             % string color
  tabsize=4,                       % sets default tabsize
}


%Function that calls listing of specific file with custom coloring
  %first input is filename to list
  %next two inputs are start and end line numbers
    %(use 0 and large number for all lines)
\newcommand\mylisting[3]{
  \tcbinputlisting{
        listing file=#1,
        breakable,
        nobeforeafter,
        arc=0pt,
        top=0mm,
        bottom=0mm,
        left=0mm,
        right=0mm,
        boxrule=0pt,
        colback=\backclr, %background color (continue accross page breaks)
        colupper=\txtupclr, %basic text color (continue accross page breaks)
        listing only,
        listing options={
                            style=mystyle,            %use custom style
                            firstline=#2,lastline=#3, %lines of code to use
                            firstnumber=#2,           %same numbering as script
                        },
        % hbox %Turning this on prevents page breaks
      }
}



%%%%%%%%%%%%%%%%%%%%%%%%%%%%%%%%%%%%%%%%%%%%%%%%%%%%%%%%%%%%%%%%%%%%%%%%
%%% MAIN DOCUMENT %%%%%%%%%%%%%%%%%%%%%%%%%%%%%%%%%%%%%%%%%%%%%%%%%%%%%%
%%%%%%%%%%%%%%%%%%%%%%%%%%%%%%%%%%%%%%%%%%%%%%%%%%%%%%%%%%%%%%%%%%%%%%%%

\title{MAE 298 -- Homework 1\\Computation of Sound Pressure Level\\and Octave Band Spectrum}

%%% first author
\author{Logan Halstrom
    \affiliation{
	PhD Graduate Student Researcher\\
	Center for Human/Robot/Vehicle Integration and Performance\\
	Department of Mechanical and Aerospace Engineering\\
	University of California, Davis\\
	% Davis, California 95616\\
    Email: ldhalstrom@ucdavis.edu
    }
}

\begin{document}
\maketitle


\newcommand\pictype{png}  %set filetype of images to read in
\newcommand\picdir{../Plots} %directory images are located in



%%%%%%%%%%%%%%%%%%%%%%%%%%%%%%%%%%%%%%%%%%%%%%%%%%%%%%%%%%%%%%%%%%%%%%
\section{Background}
%%%%%%%%%%%%%%%%%%%%%%%%%%%%%%%%%%%%%%%%%%%%%%%%%%%%%%%%%%%%%%%%%%%%%%%%

This assignment details the process of analyzing a recorded sound signal and computing the Sound Pressure Level (SPL) in Decibels (dB) accross the narrow, 1/3 Octave, and Octave-bands.  The source signal is a recording of a sonic boom contained in the included data file `Boom\_F1B2\_6.wav'.

All computations and plotting for this project were performed using Python, and the source code is attached in the Appendix.  All of the primary data processing code is contained in the file `hw1\_00\_processing.py' and the plotting code is contained in the file `hw1\_01\_plotting.py', which is suplimented by the custom plotting package `lplot.py'.

%%%%%%%%%%%%%%%%%%%%%%%%%%%%%%%%%%%%%%%%%%%%%%%%%%%%%%%%%%%%%%%%%%%%%%%%
\section{Problem 1 -- Signal Processing}
%%%%%%%%%%%%%%%%%%%%%%%%%%%%%%%%%%%%%%%%%%%%%%%%%%%%%%%%%%%%%%%%%%%%%%%%

The pressure signal input file `Boom\_F1B2\_6.wav' was read using the `PySoundFile' audio library for Python, which can be found at: \textcolor{\linkcolor}{\url{https://pypi.python.org/pypi/SoundFile/}}.  This library was chosen over other Python audio libraries (e.g. `scipy.io.wavfile') because it normalized the .wav file data between -1.0 and 1.0 identically to MATLAB's `audioread' function.  (Note: Though other libraries did not read in the same values, all libraries returned the same results when normalized by the maximum value of the data).


%%%%%%%%%%%%%%%%%%%%%%%%%%%%%%%%%%%%%%%%%%%%%%%%%%%%%%%%%%%%%%%%%%%%%%%%
\subsection{Problem 1.1 -- Sonic Boom Pressure Signal}

After reading the raw data from the .wav file, signal values were converted from units of Volts (V) to Pascals (Pa) using the conversion ratio obtained from Eqn~\ref{V2Pa}:

\begin{equation} \label{V2Pa}
-116Pa = 1V
\end{equation}

The resulting pressure history of the sonic boom is displayed in Fig~\ref{boom}, which demonstrates the classic ``N'' shape of the sonic boom, where there is an initial positive pressure discontinutity caused by the supersonic front of the aircraft, followed by an almost linear decrease in pressure over the surface of the aircraft until an abrupt discontinuity back to freestream pressure occurs.  This results in the observer hearing a ``double-boom'' for low-flying supersonic aircraft (at higher altitudes, the two shocks can merge by the time they reach the observer).  After the aircraft, the pressure does not immediately return to steady-state freestream flow, but experiences a slight oscillation in pressure as the remaining wake dissipates.

%%\vspace{-2em}
\begin{figure}[htb]
\begin{center}
\includegraphics[width=0.5\textwidth]{\picdir/1_1_Pressure.\pictype}
\caption{Recorded sonic boom shockwave pressure time history with peak pressure and N-wave time duration values (Zero-pressure from recording start to initial shock)}
\label{boom}
\end{center}
\end{figure}
%%\vspace{-2em}

Also contained in Fig~\ref{boom} are the values for the peak sonic boom pressure perturbation:

\begin{gather*}
P_{max}=\text{max}\left(|P|\right)=\boxed{75.25Pa}
\end{gather*}

\noindent and sonic boom N-wave duration:

\begin{gather*}
t_{Nwave}=\boxed{133.9ms}
\end{gather*}



%%%%%%%%%%%%%%%%%%%%%%%%%%%%%%%%%%%%%%%%%%%%%%%%%%%%%%%%%%%%%%%%%%%%%%%%
\section{Problem 1.2 -- Power Spectral Density Decomposition}

In order to analyze the frequency-dependent nature of the recorded sonic boom, it is nessesary to break the signial into its component frequencies, for which we employ the technique of Fast-Fourer Transform (FFT).  This descrete Fourier transform algorithm transforms data from the time domain to the frequency domain, and can be called in Python from `numpy.fft'.

The FFT is performed in Python according to the following pseudocode:

\begin{gather*}
fft = \text{np.fft.fft}(pressure)\cdot dt
\end{gather*}
\noindent where \emph{pressure} is the time-domain pressure signal and $dt=\dfrac{1}{f_s}$ is the time step of the discrete frequency domain, which is equal to the inverse of the sampling frequency $f_s$.

\noindent Next, the double-sided power spectrum $S_{xx}$ is obtained according to Eqn~\ref{Sxx}:

\begin{equation} \label{Sxx}
S_{xx} = \frac{|fft|^2}{T}
\end{equation}
\noindent where all operations are performed element-wise on the data series $fft$ and $T=\text{len}(fft)\cdot dt$ is the total time interval of the data series.

From the double-sided power spectrum, the singe-sided power spectrum $G_{xx}$ can be acquired as twice of the first half of $S_{xx}$ (Eqn~\ref{Gxx}):

\begin{equation} \label{Gxx}
G_{xx} = 2S_{xx}[idx]
\end{equation}
\noindent where all operations are performed element-wise on $S_{xx}$ and $idx$ is the first half of all of the indices of $S_{xx}$.

Now the the power spectral density $G_{xx}$ has been computed, the corresponding single-sided frequency spectrum can be calculated with the built in numpy function `fftfreq':

\begin{align*}
freqs &= \text{np.fft.fftfreq}(pressure{}\text{.size}, dt) \\
freqs &= freqs[idx]
\end{align*}


%%\vspace{-2em}
\begin{figure}[htb]
\begin{center}
\includegraphics[width=0.5\textwidth]{\picdir/1_2_PowerSpec.\pictype}
\caption{Shockwave signal power spectral density as a function of frequency (All frequencies above 50Hz very low power)}
\label{PowSpec}
\end{center}
\end{figure}
%%\vspace{-2em}

The resulting power spectrum $G_{xx}$ vs $freq$ is plotted for reference in Fig~\ref{PowSpec}.  From the figure, it can be seen that the most dominate frequencies occur between 1 and 50 Hz, with the maximum power spectral density of $66 Pa^2\text{/}Hz$ corresponding to $4.49Hz$, and the other significant peaks (of significantly lower magnitude) occuring around 15, 23, and 37 Hz, respectively.

%%%%%%%%%%%%%%%%%%%%%%%%%%%%%%%%%%%%%%%%%%%%%%%%%%%%%%%%%%%%%%%%%%%%%%%%
\subsection{Problem 1.3 -- Sound Pressure Level}

With the power spectrum density function calculated, it is meaninful to compute the Sound Pressure Level (SPL) in dB.  The formula for descrete SPL as a function of freqency is given in Eqn~\ref{SPL}, which demonstrates that SPL is a logarithmic ratio of the signal\textquotesingle s pressure distrubance and a reference pressure $P_{ref}=20\mu Pa$.

\begin{equation} \label{SPL}
SPL(f) = 10\log_{10}\left( \dfrac{G_{xx} / T}{P_{ref}^2} \right)
\end{equation}

\noindent SPL of the narrow-band frequencies is plotted with the octave-band frequencies in Fig~\ref{Bands} in Section~\ref{sect_oct}


%%%%%%%%%%%%%%%%%%%%%%%%%%%%%%%%%%%%%%%%%%%%%%%%%%%%%%%%%%%%%%%%%%%%%%%%
\section{Problem 2 -- Octave-Band Spectra} \label{sect_oct}
%%%%%%%%%%%%%%%%%%%%%%%%%%%%%%%%%%%%%%%%%%%%%%%%%%%%%%%%%%%%%%%%%%%%%%%%

In aeroacoustic analysis, it is common for recorded data sets to contain massive amounts of data that can be redundant and combersome to process.  To ease this burden, originally sampled narrow-band frequency spectra can be binned into fewer, pre-selected, descrete sets of freqencies called the Third (1/3) Octave and Octave Bands.

This binning process is accomplished by summing the narrow-band frequencies over intervals defined by specific center frequencies $f_c$ of the octave-band.  Center frequencies can be chosen from a pre-selected ``preferred'' set of ``nicer'' numbers, or they can be calculated according to Eqn~\ref{CenterFreqs}, which does not produce the ``nice'' numbers.

\begin{equation} \label{CenterFreqs}
f_{c,m} = f_{c,30}\cdot 2^{-10 + \frac{m}{3}}
\end{equation}
\noindent where $f_{c,30}=1000Hz$ is the center frequency corresponding to $m=30$, and $m=1,2,3,...$ for 1/3 octave-band and $m=3,6,9,...$ for octave-band.

In the data processing script for this analysis, center frequencies are calculated specifically for the narrow-band input.  The function `OctaveCenterFreqs' will loop through $m$ in intervals appropriate to the given octave-band choice, but only center frequencies whose lower $f_l$ and upper $f_u$ band limits are contained within the original data set.  This ensures that no sum will take place over an incomplete band.  Upper and lower band limits for a given center frequencies are calculated according to Eqn~\ref{BandLimits}.

\begin{equation} \label{BandLimits}
\begin{split}
f_u &= 2^{\frac{octv}{2}} \cdot f_c \\
f_l &= 2^{-\frac{octv}{2}} \cdot f_c
\end{split}
\end{equation}
\noindent where $octv=\frac{1}{3}$ for the 1/3 octave-band and $octv=1$ for the octave-band.

Once the center frequencies and associated band limits have been calculated, the sum in Eqn~\ref{OctaveLp} must be performed for each band to determine the octave-band SPL associated with each center frequency.  To calculate the 1/3 or full octave-bands from the narrow-band, simply select the center frequency bands for the desired octave and apply them in Eqn~\ref{OctaveLp}.

\begin{equation} \label{OctaveLp}
Lp(f_c) = 10\log_{10}\left( \sum_{f=f_{l,c}}^{f_{u,c}} 10^{\frac{Lp(f)}{10}} \right)
\end{equation}
\noindent where $f_{l,c},\,f_{u,c}$ are the bounds corresponding to the given center frequency $f_c$.

Finally, it can be beneficial to summarize the frequency spectrum with a single representative value called the Overall Sound Pressure Level $SPL_{ovr}$.  This parameter is calculated in the same manner as the 1/3 and full octave-band SPL, but the sum is taken over the entire data set, rather than in bins.  It is also convention not to include SPL values corresponding to less than $10Hz$, so these are ommitted in this analysis.

The results for all parts of Problem 2 are summarized below in Fig~\ref{Bands}, and they will be individually discussed in the following sections.

%%\vspace{-2em}
\begin{figure}[htb]
\begin{center}
\includegraphics[width=0.5\textwidth]{\picdir/2_SPLF_all.\pictype}
\caption{Shockwave signal narrow-band (Nar), 1/3 octave-band (Oct$\frac{1}{3}$), and octave-band (Oct), with overall Sound Pressure Level (Ovr) reported in upper right ($L_p$)}
\label{Bands}
\end{center}
\end{figure}
%%\vspace{-2em}


%%%%%%%%%%%%%%%%%%%%%%%%%%%%%%%%%%%%%%%%%%%%%%%%%%%%%%%%%%%%%%%%%%%%%%%%
\subsection{Problem 2.1 -- 1/3 Octave Bands}

Results for the 1/3 octave-band are plotted as a green line with diamond markers in Fig~\ref{Bands} and are also tabulated in Appendix A.  Compared to the narrow-band SPL (blue line), it is apparent that the binned values of the 1/3 octave-band are generally greater in magnitude than their narrow-band counterparts.  This is because the octave band is calculated as a binned sum over bands of the narrow-band and therefore \emph{must} be greater in magnitude.

The 1/3 octave-band plot follows the same trends as that of the narrow-band, but with many fewer points (43 vs. 32768, to be exact).  This demonstrates the power of using the octave bands, namely in that the general trends of the frequency spectrum can be accurately with a very few amount of points.

%%%%%%%%%%%%%%%%%%%%%%%%%%%%%%%%%%%%%%%%%%%%%%%%%%%%%%%%%%%%%%%%%%%%%%%%
\subsection{Problem 2.2 -- Octave Bands}

Results for the octave-band are plotted as a purple line with triangle markers in Fig~\ref{Bands} and are also tabulated in Appendix A. Trends are similar to those of the 1/3 octave-band, but are slightly greater in magnitude since the summing bands are wider.  Thus, the octave-band trends are less accurate at modeling the narrow-band spectrum than the 1/3 octave-band, with the advantage of having even fewer points (14 vs. 43 vs. 32768 for the octave, 1/3 octave, and narrow-bands, respectively).


%%%%%%%%%%%%%%%%%%%%%%%%%%%%%%%%%%%%%%%%%%%%%%%%%%%%%%%%%%%%%%%%%%%%%%%%
\subsection{Problem 2.3 -- Overall Sound Pressure Level}

Finally, the overall SPL of the sonic boom signal was found to be $\boxed{SPL_{ovr}=111.33dB}$, and is represented in Fig~\ref{Bands} as a dashed, black line.  This value is generally greater in magnitude than the majority of the points in any of the spectra, especially at high frequency.  Each of the narrow, 1/3, and octave-bands peak at a higher value than $SPL_{ovr}$, however, because $SPL_{ovr}$ does not include information for frequencies below $10Hz$.





%%%%%%%%%%%%%%%%%%%%%%%%%%%%%%%%%%%%%%%%%%%%%%%%%%%%%%%%%%%%%%%%%%%%%%%%
\section{Conclusion}

In summary, this analysis has demonstrated basic signal processing techniques required to analyze aeroacoustical data.  Reading raw pressure time histories and transformation into the descrete frequency domain was demonstrated in Problem 1, and conversion into the octave-bands and overall SPL was performed in Problem 2.  These techniques will be required for future, more advanced aeroacoustic analyses.

%%%%%%%%%%%%%%%%%%%%%%%%%%%%%%%%%%%%%%%%%%%%%%%%%%%%%%%%%%%%%%%%%%%%%%%%
% \section{References}
% \begin{description}
% \item[] 1. Bogard, D.G., Teiderman, W.G. ``Burst detection with single-point velocity measurements'', \emph{Journal of Fluid Mechanics}, 162:389-413, 1986.
% \end{description}








%%%%%%%%%%%%%%%%%%%%%%%%%%%%%%%%%%%%%%%%%%%%%%%%%%%%%%%%%%%%%%%%%%%%%%%%
%APPENDIX (CODE LISTING)

% \clearpage
\onecolumn      %change to single column
\appendix       %%% starting appendix

%SETUP
\newcommand\scriptname{dummy}      %init code name variable
\captionsetup{width=0.8\linewidth} %set caption text width

\catcode`\_=12 %ESCAPE ALL UNDERSCORES IN TEXT AUTOMATICALLY*******************

%LISTINGS

\renewcommand\scriptname{hw1_00_process.py}
\section*{Appendix A: Data Processing Script}
\begin{center}
\mylisting{../\scriptname}{0}{10000}
% \vspace{-0.2cm}
\captionof{lstlisting}{\emph{\scriptname} - Performs all primary data processing such as pressure signal input, power spectral density decomposition, and octave-band conversion and saves data to text files}
\end{center}

\clearpage

% \renewcommand\scriptname{hw1_99_wrapper.py}
% \section*{Appendix C: Data Processing/Plotting Wrapper Script}
% \begin{center}
% \mylisting{../\scriptname}{0}{10000}
% % \vspace{-0.2cm}
% \captionof{lstlisting}{\emph{\scriptname} - Wrapper program that runs data processing and plotting scripts simultaneously}
% \end{center}

% \clearpage



% \section*{Appendix C: Parallel, Multi-Block Grid Decomposition Code}
% \lstinputlisting[caption=Grids are decomposed into blocks and mapped onto NPROCS processors and information pertaining to neighbors is stored using the GRIDMOD module, language=Fortran]{../modules.f90}

% \clearpage

% \section*{Appendix D: Multi-Block Plot3D Reader-Writer}
% \lstinputlisting[caption=Code for saving formatted multiblock PLOT3D solution files and reading restart files, language=Fortran]{../inout.f90}

% \clearpage




%%%%%%%%%%%%%%%%%%%%%%%%%%%%%%%%%%%%%%%%%%%%%%%%%%%%%%%%%%%%%%%%%%%%%%
%\clearpage


%%%%%%%%%%%%%%%%%%%%%%%%%%%%%%%%%%%%%%%%%%%%%%%%%%%%%%%%%%%%%%%%%%%%%%
% The bibliography is stored in an external database file
% in the BibTeX format (file_name.bib).  The bibliography is
% created by the following command and it will appear in this
% position in the document. You may, of course, create your
% own bibliography by using thebibliography environment as in
%
% \begin{thebibliography}{12}
% ...
% \bibitem{itemreference} D. E. Knudsen.
% {\em 1966 World Bnus Almanac.}
% {Permafrost Press, Novosibirsk.}
% ...
% \end{thebibliography}

% Here's where you specify the bibliography style file.
% The full file name for the bibliography style file
% used for an ASME paper is asmems4.bst.
%\bibliographystyle{asmems4}

% Here's where you specify the bibliography database file.
% The full file name of the bibliography database for this
% article is asme2e.bib. The name for your database is up
% to you.
%\bibliography{asme2e}

%%%%%%%%%%%%%%%%%%%%%%%%%%%%%%%%%%%%%%%%%%%%%%%%%%%%%%%%%%%%%%%%%%%%%%


\end{document}
