
%%% use twocolumn and 10pt options with the asme2ej format
\documentclass[twocolumn,10pt]{asme2ej}
%% The class has several options
%  onecolumn/twocolumn - format for one or two columns per page
%  10pt/11pt/12pt - use 10, 11, or 12 point font
%  oneside/twoside - format for oneside/twosided printing
%  final/draft - format for final/draft copy
%  cleanfoot - take out copyright info in footer leave page number
%  cleanhead - take out the conference banner on the title page
%  titlepage/notitlepage - put in titlepage or leave out titlepage
%
%% The default is oneside, onecolumn, 10pt, final


\usepackage{epsfig} %% for loading postscript figures
\usepackage{listings}
\usepackage{amsmath}
\usepackage{graphicx}
\usepackage{grffile}
\usepackage{pdfpages}
\usepackage{algpseudocode}
\usepackage{courier}
\usepackage{tikz}
\newcommand*\circled[1]{\tikz[baseline=(char.base)]{
            \node[shape=circle,draw,inner sep=2pt] (char) {#1};}}
%\usepackage{multicol}


% Custom colors
\usepackage{color}
\usepackage{listings}
\usepackage{framed}
\usepackage{caption}
\usepackage{bm}
\captionsetup[lstlisting]{font={small,tt}}

\definecolor{mygreen}{rgb}{0,0.6,0}
\definecolor{mygray}{rgb}{0.5,0.5,0.5}
\definecolor{mymauve}{rgb}{0.58,0,0.82}

\lstset{ %
  backgroundcolor=\color{white},   % choose the background color; you must add \usepackage{color} or \usepackage{xcolor}
  basicstyle=\ttfamily\footnotesize, % the size of the fonts that are used for the code
  breakatwhitespace=false,         % sets if automatic breaks should only happen at whitespace
  % breaklines=true,                 % sets automatic line breaking
  captionpos=b,                    % sets the caption-position to bottom
  commentstyle=\color{mygreen},    % comment style
  deletekeywords={...},            % if you want to delete keywords from the given language
  escapeinside={\%*}{*)},          % if you want to add LaTeX within your code
  extendedchars=true,              % lets you use non-ASCII characters; for 8-bits encodings only, does not work with UTF-8
  frame=single,                    % adds a frame around the code
  keepspaces=true,                 % keeps spaces in text, useful for keeping indentation of code (possibly needs columns=flexible)
  columns=flexible,
  keywordstyle=\color{blue},       % keyword style
  language=Python,                 % the language of the code
  morekeywords={*,...},            % if you want to add more keywords to the set
  numbers=left,                    % where to put the line-numbers; possible values are (none, left, right)
  numbersep=5pt,                   % how far the line-numbers are from the code
  numberstyle=\tiny\color{mygray}, % the style that is used for the line-numbers
  rulecolor=\color{black},         % if not set, the frame-color may be changed on line-breaks within not-black text (e.g. comments (green here))
  showspaces=false,                % show spaces everywhere adding particular underscores; it overrides 'showstringspaces'
  showstringspaces=false,          % underline spaces within strings only
  showtabs=false,                  % show tabs within strings adding particular underscores
  stepnumber=1,                    % the step between two line-numbers. If it's 1, each line will be numbered
  stringstyle=\color{mymauve},     % string literal style
  tabsize=4,                       % sets default tabsize to 2 spaces
}





\title{MAE 298 -- Homework 1\\Computation of Sound Pressure Level\\and Octave Band Spectrum}

%%% first author
\author{Logan Halstrom
    \affiliation{
	PhD Graduate Student Researcher\\
	Center for Human/Robot/Vehicle Integration and Performance\\
	Department of Mechanical and Aerospace Engineering\\
	University of California, Davis\\
	% Davis, California 95616\\
    Email: ldhalstrom@ucdavis.edu
    }
}

\begin{document}
\maketitle


\newcommand\pictype{png}  %set filetype of images to read in
\newcommand\picdir{../Plots} %directory images are located in



%%%%%%%%%%%%%%%%%%%%%%%%%%%%%%%%%%%%%%%%%%%%%%%%%%%%%%%%%%%%%%%%%%%%%%
\section{Introduction}
%%%%%%%%%%%%%%%%%%%%%%%%%%%%%%%%%%%%%%%%%%%%%%%%%%%%%%%%%%%%%%%%%%%%%%%%

Give overview of homework and background concepts


%%%%%%%%%%%%%%%%%%%%%%%%%%%%%%%%%%%%%%%%%%%%%%%%%%%%%%%%%%%%%%%%%%%%%%%%
\section{Read Data}
%%%%%%%%%%%%%%%%%%%%%%%%%%%%%%%%%%%%%%%%%%%%%%%%%%%%%%%%%%%%%%%%%%%%%%%%

list functions used to read data, how python/matlab compare.

%%\vspace{-2em}
\begin{figure}[htb]
\begin{center}
\includegraphics[width=0.5\textwidth]{\picdir/1_1_Pressure.\pictype}
\caption{Recorded sonic boom shockwave pressure time history in characteristic high-low pressure N-wave shape (Zero-pressure from recording start to initial shock)}
\label{pic_Nwave}
\end{center}
\end{figure}
%%\vspace{-2em}



%%%%%%%%%%%%%%%%%%%%%%%%%%%%%%%%%%%%%%%%%%%%%%%%%%%%%%%%%%%%%%%%%%%%%%%%
\section{Frequency Domain}
%%%%%%%%%%%%%%%%%%%%%%%%%%%%%%%%%%%%%%%%%%%%%%%%%%%%%%%%%%%%%%%%%%%%%%%%

decompose into frequency domain with FFT


%%%%%%%%%%%%%%%%%%%%%%%%%%%%%%%%%%%%%%%%%%%%%%%%%%%%%%%%%%%%%%%%%%%%%%%%
\subsection{Power Spectral Density Decomposition}

power spectrum density decomposition stuff

%%\vspace{-2em}
\begin{figure}[htb]
\begin{center}
\includegraphics[width=0.5\textwidth]{\picdir/1_2_PowerSpec.\pictype}
\caption{Shockwave signal power spectral density as a function of frequency (All frequencies above 50Hz very low power)}
\label{pic_spec}
\end{center}
\end{figure}
%%\vspace{-2em}

%%%%%%%%%%%%%%%%%%%%%%%%%%%%%%%%%%%%%%%%%%%%%%%%%%%%%%%%%%%%%%%%%%%%%%%%
\subsection{Sound Pressure Level}

this is actually in the plot in the next section


%%%%%%%%%%%%%%%%%%%%%%%%%%%%%%%%%%%%%%%%%%%%%%%%%%%%%%%%%%%%%%%%%%%%%%%%
\section{Octave-Band Spectra} \label{sect_oct}
%%%%%%%%%%%%%%%%%%%%%%%%%%%%%%%%%%%%%%%%%%%%%%%%%%%%%%%%%%%%%%%%%%%%%%%%

%%\vspace{-2em}
\begin{figure}[htb]
\begin{center}
\includegraphics[width=0.5\textwidth]{\picdir/2_SPLF_all.\pictype}
\caption{Shockwave signal narrow-band, $\frac{1}{3}$ octave-band, and octave-band, with overall Sound Pressure Level reported in upper right}
\label{pic_oct}
\end{center}
\end{figure}
%%\vspace{-2em}




%%%%%%%%%%%%%%%%%%%%%%%%%%%%%%%%%%%%%%%%%%%%%%%%%%%%%%%%%%%%%%%%%%%%%%%%
\section{Conclusion}

conclude

%%%%%%%%%%%%%%%%%%%%%%%%%%%%%%%%%%%%%%%%%%%%%%%%%%%%%%%%%%%%%%%%%%%%%%%%
% \section{References}
% \begin{description}
% \item[] 1. Bogard, D.G., Teiderman, W.G. ``Burst detection with single-point velocity measurements'', \emph{Journal of Fluid Mechanics}, 162:389-413, 1986.
% \end{description}








%%%%%%%%%%%%%%%%%%%%%%%%%%%%%%%%%%%%%%%%%%%%%%%%%%%%%%%%%%%%%%%
% \clearpage
\onecolumn
\appendix       %%% starting appendix

% \section*{Appendix A: Parallel, Multi-Block Grid Wrapper Code}
% \lstinputlisting[caption=Wrapper program that performs domain decomposition on a single processor and then independently reads in solution for solver, language=Fortran]{../main.f90}

% \clearpage

% \section*{Appendix B: Parallel, Multi-Block Grid Solver Code}
% \lstinputlisting[caption=Grid decomposition and solution processess are organized in this code, language=Fortran]{../subroutines.f90}

% \section*{Appendix C: Parallel, Multi-Block Grid Decomposition Code}
% \lstinputlisting[caption=Grids are decomposed into blocks and mapped onto NPROCS processors and information pertaining to neighbors is stored using the GRIDMOD module, language=Fortran]{../modules.f90}

% \clearpage

% \section*{Appendix D: Multi-Block Plot3D Reader-Writer}
% \lstinputlisting[caption=Code for saving formatted multiblock PLOT3D solution files and reading restart files, language=Fortran]{../inout.f90}

% \clearpage




%%%%%%%%%%%%%%%%%%%%%%%%%%%%%%%%%%%%%%%%%%%%%%%%%%%%%%%%%%%%%%%%%%%%%%
%\clearpage


%%%%%%%%%%%%%%%%%%%%%%%%%%%%%%%%%%%%%%%%%%%%%%%%%%%%%%%%%%%%%%%%%%%%%%
% The bibliography is stored in an external database file
% in the BibTeX format (file_name.bib).  The bibliography is
% created by the following command and it will appear in this
% position in the document. You may, of course, create your
% own bibliography by using thebibliography environment as in
%
% \begin{thebibliography}{12}
% ...
% \bibitem{itemreference} D. E. Knudsen.
% {\em 1966 World Bnus Almanac.}
% {Permafrost Press, Novosibirsk.}
% ...
% \end{thebibliography}

% Here's where you specify the bibliography style file.
% The full file name for the bibliography style file
% used for an ASME paper is asmems4.bst.
%\bibliographystyle{asmems4}

% Here's where you specify the bibliography database file.
% The full file name of the bibliography database for this
% article is asme2e.bib. The name for your database is up
% to you.
%\bibliography{asme2e}

%%%%%%%%%%%%%%%%%%%%%%%%%%%%%%%%%%%%%%%%%%%%%%%%%%%%%%%%%%%%%%%%%%%%%%


\end{document}
