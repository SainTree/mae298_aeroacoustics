
\typeout{}\typeout{If latex fails to find aiaa-tc, read the README file!}
%


\documentclass[]{aiaa-tc}% insert '[draft]' option to show overfull boxes

\usepackage{mathptmx}         %CHANGE FONT TO TIMES NEW ROMAN

\usepackage{amsmath}          % for formula writing (i.e. 'split', etc)
\usepackage{rotate}           %rotate/mirror images
\usepackage{cancel}           %draw lines through math to show "goes to zero"
\usepackage{xfrac}            %allows slated and side fractions
\usepackage{subcaption}       %allows captioning individual subfigures
\usepackage{multicol}         %enable environment with multiple columns
\usepackage[mode=buildnew]{standalone}% requires -shell-escape
  % compile with `pdflatex -shell-escape main` or `xelatex  -shell-escape main`


\usepackage{tikz}             %for creating vector graphics diagrams
\usetikzlibrary{backgrounds}  %put backgrounds behind tikz figures
\usetikzlibrary{calc}         %perform calculations within $$
\usetikzlibrary{positioning}  %position tikz elements using "right of, etc"
\usetikzlibrary{angles}       %label angles between lines with arcs
\usetikzlibrary{quotes}       %Put angle label in quotes
\usetikzlibrary{patterns}     %Patterns to fill shapes with






  \title{MAE 298 Aeroacoustics -- Final Project \\ Aeroacoustics of Aerospace Vehicles During Launch and Ascent}

\author{
  Logan D. Halstrom \\
  {\normalsize\itshape Graduate Student} \\
  {\normalsize\itshape Department of Mechanical and Aerospace Engineering} \\
  {\normalsize\itshape University of California, Davis, CA 95616}
       }


 % Define commands to assure consistent treatment throughout document
 \newcommand{\eqnref}[1]{(\ref{#1})}
 \newcommand{\class}[1]{\texttt{#1}}
 \newcommand{\package}[1]{\texttt{#1}}
 \newcommand{\file}[1]{\texttt{#1}}
 \newcommand{\BibTeX}{\textsc{Bib}\TeX}

%%%%%%%%%%%%%%%%%%%%%%%%%%%%%%%%%%%%%%%%%%%%%%%%%%%%%%%%%%%%%%%%%%%%%%%%
\begin{document}

\maketitle

% %%%%%%%%%%%%%%%%%%%%%%%%%%%%%%%%%%%%%%%%%%%%%%%%%%%%%%%%%%%%%%%%%%%%%%%%
\begin{abstract}

Abstract about lit study

\end{abstract}

%%%%%%%%%%%%%%%%%%%%%%%%%%%%%%%%%%%%%%%%%%%%%%%%%%%%%%%%%%%%%%%%%%%%%%%%
\section*{Nomenclature}

\begin{multicols}{2}

\begin{tabbing}
  XXX \= \kill% this line sets tab stop
  $0$                 \> Subscript for quiescent parameters \\
  $e$                 \> Subscript for emission parameters \\
  $L$                 \> Subscript for loading parameters \\
  $t$                 \> Time \\
  $\tau$              \> Retarded time \\
  $ret$               \> Evaluated at retarded time \\
  $\hat{n}$           \> Unit surface normal vector \\
  $\vec{r}$           \> Radial direction vector \\
  $\hat{r}$           \> Radial unit vector \\
  $r$                 \> Magnitude of radial vector $|\vec{r}|$ \\
  $\theta$            \> Angle between $\hat{n}$ and $\hat{r}$ \\
  $f$                 \> Function of surfaces within a fluid space \\
  $\vec{V}$           \> General velocity vector \\
  $V_r$               \> Velocity component in radial direction \\
  $V_n$               \> Velocity component in surface normal direction \\
  $M$                 \> Mach number \\
  $c$                 \> Speed of Sound\\
  $\overline{\rho}$   \> Mean density \\
  $p$                 \> Pressure  \\
  $\widetilde{p}$     \> Pressure (Discontinuous across data surface)  \\
  $\overline{p}$      \> Mean pressure \\
  $p'$                \> Perturbation pressure \\
  $\Delta P$          \> Pressure difference from CFD solution \\
  $L$                 \> Pressure loading \\
  $\overline{\partial}$ \> Generalized derivative \\
  $\delta$            \> Dirac delta function \\
  \scriptsize{FW-H}   \> Ffowcs Williams-Hawkings\\






\end{tabbing}

\end{multicols}

%%%%%%%%%%%%%%%%%%%%%%%%%%%%%%%%%%%%%%%%%%%%%%%%%%%%%%%%%%%%%%%%%%%%%%%%
\section{Introduction} %%%%%%%%%%%%%%%%%%%%%%%%%%%%%%%%%%%%
%%%%%%%%%%%%%%%%%%%%%%%%%%%%%%%%%%%%%%%%%%%%%%%%%%%%%%%%%%%%%%%%%%%%%%%%

In this assignment, we will first derive the wave equation for sound generated by a moving body using generalized differentiation.  This process is similar to that used by Ffowcs Williams and Hawkings in the derivation of their equation describing aeroacoustic noise.  Equations like these are useful for computing aeroacoustic effects once source terms such as body loading are calculated.

Secondly, we will derive Farassat's Formulation 1A of the Ffowcs Williams-Hawkings (FW-H) equation for pressure due to loading noise beginning with Formulation 1.  This result will demonstrate the process for solving the FW-H equation and may be applied to the other terms to obtain Farassat's complete Formulation 1A of the FW-H equation.





%%%%%%%%%%%%%%%%%%%%%%%%%%%%%%%%%%%%%%%%%%%%%%%%%%%%%%%%%%%%%%%%%%%%%%%%
\section{Main Discussion}
%%%%%%%%%%%%%%%%%%%%%%%%%%%%%%%%%%%%%%%%%%%%%%%%%%%%%%%%%%%%%%%%%%%%%%%%

In this section, we will derive the wave equation for sound generated by a moving body, which is known as the Kirchhoff formula for moving surfaces.  The general acoustic wave equation with no source term (homogeneous) is expressed as follows:

\begin{equation} \label{AcousticWaveHomo}
\boxed{\dfrac{1}{c^2}\dfrac{\partial^2p}{\partial t^2} - \nabla^2p = 0}
\end{equation}

We will used generalized differentiation to show that the wave equation whose sound is generated by an arbitrary moving body $f=0$ can be expressed as follows:

\begin{equation} \label{AcousticWaveBody}
\boxed{\dfrac{1}{c^2}\dfrac{ \overline{\partial}^2\widetilde{p}}{\partial t^2}
    - \overline{\nabla}^2 \widetilde{p}
= -\left[ \dfrac{M_n}{c} \dfrac{\partial p}{\partial t} + p_n \right] \delta(f)
    -\dfrac{1}{c} \dfrac{\partial}{\partial t} \left[ M_n p \delta(f) \right]
    -\nabla \cdot \left[ p \hat{n} \delta(f) \right]}
\end{equation}

\noindent where $\hat{n}$ is the unit normal vector on the surface and $p_n = \nabla p \cdot \hat{n}$.

We will use Green’s function of the wave equation in the unbounded space to find the unknown function $p(\vec{x},t)$, which exists everywhere in space. This result is called the Kirchhoff formula for moving surfaces.


%%%%%%%%%%%%%%%%%%%%%%%%%%%%%%%%%%%%%%%%%%%%%%%%%%%%%%%%%%%%%%%%%%%%%%%%
\subsection{Sub topic}

The moving surface generating acoustic effects will be defined as $f(\vec{x},t) = 0$ and will be referred to as a ``data surface''.  The definition of $f$ allows the fluid to be defined at all points in space, which can be divided into three regions:

\begin{center}
\begin{tabular}{| r  l |}
  \hline
  Data Surface:    & $f=0$ \\
  Exterior Region: & $f>0$ \\
  Interior Region: & $f>0$ \\
  \hline
\end{tabular}
\end{center}

\noindent We will also assume that the gradient of $f$ is the surface outward unit normal vector $\hat{n}$:

\begin{equation} \label{Gradfnormal}
\nabla f = \hat{n}
\end{equation}

Under this definition of $f$, the fluid will exist in all regions of unbounded space and thus enable the usage of Green's function to solve for the sound field.

We will solve for the sound field in the surface exterior region, but we will assume that the fluid that extends into the interior region has the same conditions as an undisturbed, quiescent medium.  We can define a pressure variable with these properties as $\widetilde{p}$ using the embedding technique as follows:

\begin{equation} \label{Ptilde}
\begin{split}
\boxed{\widetilde{p} =
    \left\{ \begin{array}{lll}
        p, & f > 0 \\
        0, & f < 0
    \end{array} \right.}
\end{split}
\end{equation}

\noindent Thus, pressure will vary outside of the data surface and will be undisturbed inside. Though not continuous, this function of pressure will be defined at all points in an unbounded space and thus applicable to Green's function.






%%%%%%%%%%%%%%%%%%%%%%%%%%%%%%%%%%%%%%%%%%%%%%%%%%%%%%%%%%%%%%%%%%%%%%%%
\section{Conclusions}
%%%%%%%%%%%%%%%%%%%%%%%%%%%%%%%%%%%%%%%%%%%%%%%%%%%%%%%%%%%%%%%%%%%%%%%%

We have now completed the derivation of an aeroacoustic equation describing the pressure fluctuations around a moving body and then demonstrated a method of solving equations such as these.  In a real life application, the equation being solved would be more complicated and include more source terms, like the complete FW-H equation does.  This equation can be solved by applying Farassat's Formulation 1A and implementing the solution numerically.  Source inputs for loading noise ($\Delta P$) would be obtained from an unsteady CFD simulation and used to inform the aeroacoustic solution.


%%%%%%%%%%%%%%%%%%%%%%%%%%%%%%%%%%%%%%%%%%%%%%%%%%%%%%%%%%%%%%%%%%%%%%%%
\begin{thebibliography}{9}% maximum number of references (for label width)
%%%%%%%%%%%%%%%%%%%%%%%%%%%%%%%%%%%%%%%%%%%%%%%%%%%%%%%%%%%%%%%%%%%%%%%%

 \bibitem{ray:15}
 E. S. Ray and R. A. Machin, ``Pendulum Motion in Main Parachute Clusters,'' in {\it 23rd AIAA Aerodynamic Decelerator Systems Technology Conference}, Submitted, 2015.

 \bibitem{menter_sst}
 F. R. Menter and Christopher L. Rumsey, ``Assessment of Two-Equation Turbulence Models for Transonic Flows,'' in {\it 25th AIAA Fluid Dynamics Conference}, Submitted, 1994.

 \bibitem{schwing15}
 J. S. Greathouse and A. M. Schwing, ``Study of Geometric Porosity on Static Stability and Drag using Computational Fluid Dynamics for Rigid Parachute Shapes,'' in {\it 23rd AIAA Aerodynamic Decelerator Systems Technology Conference}, Submitted, 2015.

 \bibitem{overflow}
 R. H. Nichols, R. W. Tramel, and P. G. Buning, “Solver and Turbulence Model Upgrades to OVERFLOW 2 for Unsteady and High-Speed Applications,” No. AIAA 2006-2824, 2006.

 \bibitem{peg5}
 S. E. Rogers, K. Roth, M. Field, S. M. Nash, M. D. Baker, J. P. Slotnick, M. Whitlock, L. Beach, and H. V. Cao, ``Advances in Overset CFD Processes Applied to Subsonic High-Lift Aircraft'', No. AIAA 2000-4216, 2000.

 \bibitem{dcf}
 W. M. Chan, N. Kim, and S. A. Pandya, ``Advances in Domain Connectivity for Overset Grids Using the X-rays Approach,'' in {\it Seventh International Conference on Computational Fluid Dynamics}, Submitted, 2012.

 \bibitem{gmp}
 S. M. Murman, W. M. Chan, M. J. Aftosmis, and R. L. Meakin, ``An Interface for Specifying Rigid-Body Motions for CFD Applications,'' AIAA Paper 2003-1237, 2003.

 \bibitem{blockage}
 J. M. Macha and R. J. Buffington, ``Wall-Interference Corrections for Parachutes in a Closed Wind Tunnel,'' {\it Journal of Aircraft}, Vol. 27, No. 4, pp. 320-325, April 1990.



\end{thebibliography}





\end{document}


